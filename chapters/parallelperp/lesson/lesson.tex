% Lesson content - Parallel and Perpendicular Lines
\section{Erinnere Dich}
\subsection{\fcolorbox{violet}{orrange}{Senkrechte und parallele Geraden}}

\cadreinfo{Schnittpunkt}{
\begin{minipage}{12.5cm}
 \setlength{\columnseprule}{0pt}
\begin{multicols}{2}
Wenn zwei Geraden sich schneiden, dann hei\ss t ihren gemeinsamen Punkt \textbf{Schnittpunkt}.
\columnbreak

\begin{pspicture}(0,0)(6,1)
\psset{yunit=0.5}
\pstGeonode[PointSymbol=none, PointName=none](0,0){A}(6,3){B}(0,2.5){C}(5,0){D}
\pstLineAB{A}{B}
\pstLineAB{C}{D}
\pstInterLL[PosAngle=90]{A}{B}{C}{D}{P}
\end{pspicture}

\end{multicols}
Zwei  Geraden, die sich schneiden hei\ss en \og \textbf{sekante} Geraden \fg.
\end{minipage}
}