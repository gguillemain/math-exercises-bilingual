% !TEX root = main.tex
% Configuration file for bilingual math textbook

% Document class and basic settings
\documentclass[12pt,a4paper]{book}

% Basic encoding and font settings
\usepackage[utf8]{inputenc}
\usepackage[T1]{fontenc}
\usepackage{lmodern}
\usepackage{textcomp}

% Language support
\usepackage[french,german]{babel}
\usepackage{csquotes}

% Mathematics packages
\usepackage{amsmath,amssymb,amsfonts}
\usepackage{mathrsfs}
\usepackage{wasysym}
\usepackage{mathcomp}
\usepackage{cancel}
\usepackage{fp}

% Graphics and geometry
\usepackage{pstricks,pstricks-add}
\usepackage{pst-plot}
\usepackage{pst-eucl}
\usepackage{pst-node}
\usepackage{pst-tree}
\usepackage{pst-solides3d}
\usepackage{pst-platon}
\usepackage{auto-pst-pdf}
\usepackage{graphicx}
\usepackage{tikz}
\usepackage{pgfplots}
\pgfplotsset{compat=1.17}

% TikZ libraries
\usetikzlibrary{matrix,positioning,shapes,decorations.text,calc,decorations.pathmorphing,patterns}

% Layout and design
\usepackage[a4paper,top=1cm,left=1.5cm,bottom=1cm,right=1.5cm,heightrounded,includeheadfoot]{geometry}
\usepackage{fancyhdr}
\usepackage{multicol}
\usepackage{enumitem}
\usepackage{fancybox}
\usepackage{array}
\usepackage{tabularx}
\usepackage{multirow}
\usepackage{hhline}
\usepackage{dcolumn}
\usepackage{setspace}
\usepackage{lscape}
\usepackage{numprint}
\usepackage{xifthen}

% Title formatting
\usepackage[sf,sl,outermarks]{titlesec}
\usepackage[Glenn]{fncychap}
\usepackage{titletoc}
\usepackage{tocloft}
\usepackage{dashrule}

% Colors
\usepackage{xcolor}
\usepackage{colortbl}
\usepackage{bclogo}
\usepackage{soul}

% Define custom colors
\definecolor{jorange}{rgb}{0.179,0.628,0.898}
\definecolor{violet}{rgb}{0.695,0.07,0.558}
\definecolor{violetclair}{rgb}{0.998,0.5,1}
\definecolor{violetfonce}{rgb}{0.3,0.03,0.238}
\definecolor{vertclair}{rgb}{0.89,0.95,0.40}
\definecolor{vert}{rgb}{0.552,0.764,0.509}
\definecolor{vert1}{rgb}{0.79,1,0.43}
\definecolor{orange}{rgb}{0.179,0.628,0.898}
\definecolor{orrange}{rgb}{1,0.69,0}
\definecolor{bleuclair}{rgb}{0.468,0.867,0.998}
\definecolor{coulecon}{rgb}{0.758,0.805,0.945}

% JSON parsing for translations
\usepackage{xstring}
\usepackage{catchfile}
\usepackage{environ}

% Translation system commands
\newcommand{\loadTranslations}[1]{%
  \IfFileExists{#1}{%
    \CatchFileDef{\translations}{#1}{}%
  }{%
    \typeout{Warning: Translation file #1 not found}%
  }%
}

\newcommand{\trans}[1]{%
  % Cette commande sera remplacée par le script Python
  \textbf{#1}%
}

\newcommand{\transTitle}[1]{%
  % Cette commande sera remplacée par le script Python
  \textbf{#1}%
}

\newcommand{\transInstructions}[1]{%
  % Cette commande sera remplacée par le script Python
  \textbf{#1}%
}

% Teacher mode switch
\newif\ifteacher
\teacherfalse  % Default: student mode
%\teachertrue  % Uncomment for teacher mode

% Solution environment
\ifteacher
  \newenvironment{solution}{%
    \begin{center}
    \begin{minipage}{0.95\textwidth}
    \textbf{Solution :}\\
  }{%
    \end{minipage}
    \end{center}
  }
\else
  \NewEnviron{solution}{}  % Hide solutions in student mode
\fi

% Custom box environments
\newcommand{\cadreinfo}[2]{
\begin{bclogo}[logo=\bcinfo,couleur=coulecon, epBord=0, arrondi = 0.1, ombre = false, epOmbre = 0.15, couleurOmbre = black!30]{~\fboxsep 2pt\fcolorbox{coulecon}{cyan}{\textcolor{white}{#1}}}
#2
\end{bclogo}
}

\newcommand{\cadrecrayon}[2]{
\begin{bclogo}[logo=\bccrayon,couleur=coulecon, epBord=0, arrondi = 0.1, ombre = false, epOmbre = 0.15, couleurOmbre = black!30]{~\fboxsep 2pt\fcolorbox{coulecon}{cyan}{\textcolor{white}{#1}}}
#2
\end{bclogo}
}

\newcommand{\cadrevok}[1]{
\begin{bclogo}[logo=\bcdallemagne,couleur = cyan, arrondi = 0.1, ombre = true, epOmbre = 0.15, couleurOmbre = black!30,barre=snake,couleurBarre=white,blur]{~\textcolor{white}{Vokabeln}}
\noindent#1
\end{bclogo}
}

\newcommand{\cadretrad}[1]{
\begin{bclogo}[logo=\bcdallemagne,couleur = cyan, arrondi = 0.1,barre=snake,couleurBarre=white,blur,couleurBord=cyan]{~\textcolor{white}{Vokabeln}}
\noindent#1
\end{bclogo}
}

% Custom math symbols and environments
\newcommand{\R}{\mathbb{R}}
\newcommand{\N}{\mathbb{N}}
\newcommand{\D}{\mathbb{D}}
\newcommand{\Z}{\mathbb{Z}}
\newcommand{\Q}{\mathbb{Q}}
\newcommand{\C}{\mathbb{C}}

\newcommand{\vect}[1]{\mathchoice%
{\overrightarrow{\displaystyle\mathstrut#1\,\,}}%
{\overrightarrow{\textstyle\mathstrut#1\,\,}}%
{\overrightarrow{\scriptstyle\mathstrut#1\,\,}}%
{\overrightarrow{\scriptscriptstyle\mathstrut#1\,\,}}}

\def\Oij{$\left(\text{O},~\vect{\imath},~\vect{\jmath}\right)$}
\def\Oijk{$\left(\text{O},~\vect{\imath},~ \vect{\jmath},~ \vect{k}\right)$}
\def\Ouv{$\left(\text{O},~\vect{u},~\vect{v}\right)$}

% Index configuration
\usepackage{makeidx}
\usepackage{imakeidx}
\makeindex[program=makeindex,title=Index des mots utilisés dans le livre,columns=3,intoc=true,columnsep=10pt]

% Header and footer style
\pagestyle{fancy}
\fancyhf{}
\renewcommand{\headrulewidth}{0.4pt}
\renewcommand{\footrulewidth}{0.4pt}

% Custom commands for German math vocabulary
\newcommand{\gv}{Größenvorstellung~}

% Enumeration settings
\renewcommand{\theenumi}{\textbf{\arabic{enumi}}}
\renewcommand{\labelenumi}{\textbf{\theenumi.}}
\renewcommand{\theenumii}{\textbf{\alph{enumii}}}
\renewcommand{\labelenumii}{\textbf{\theenumii.}}

% Custom presentation commands
\newcommand\NomPrenom{\textbf{\textit{Nom :\hfill Prénom :\hfill Classe :}}\hspace*{2cm}}
\newcommand*{\titre}[1]{{\centering\bfseries\scshape\Large#1\par}}
\newcommand*{\ladate}[1]{\vspace{0.1cm}{\centering\itshape#1\par}\vspace{0.1cm}}

% Exercise formatting
\newcommand*{\ubungbox}[1]{\fboxsep 0pt\fcolorbox{yellow}{orrange}{#1}}
\newcommand*{\exo}[1]{\vspace{0.35cm plus 0.15cm minus 0.15cm}\rule{1ex}{1ex}\hspace{1ex}{\ubungbox{\textbf{#1}}}\vspace{0.1cm plus 0.1cm minus 0.1cm}}
\newcommand*{\aufgabe}[1]{\vspace{0.35cm plus 0.15cm minus 0.15cm}\rule{6ex}{1ex}\hspace{3ex}\textsc{\textbf{#1}}\vspace{0.1cm plus 0.1cm minus 0.1cm}}

% German specific commands
\newcommand{\beispiel}{\textit{\large\ding{253}Beispiel : }\smallskip}
\newcommand{\bemerkung}{\textit{\large\ding{46}Bemerkung : }\smallskip}
